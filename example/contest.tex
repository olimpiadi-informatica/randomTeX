% Template per le Selezioni Scolastiche delle OII

\documentclass[a4paper]{article}
\usepackage[utf8x]{inputenc}
\usepackage{lmodern}
\usepackage[margin=0.8in]{geometry} % il minimo sensato è 0.25in vedi qua: https://stackoverflow.com/a/3513476/747654
\usepackage[italian]{babel}
\usepackage{graphicx} % per \includegraphics{}
\usepackage{tabularx} % per tabularx, una versione di tabular che permette di "espandere" la colonna
\usepackage{textcomp} % per il simbolo \textdegree{}
\usepackage[hidelinks]{hyperref}
\usepackage{enumitem}
\usepackage{etoolbox}
\usepackage{xcolor}
\usepackage{soul}
\usepackage{amsmath}
\usepackage{float}
\usepackage[section]{algorithm}
\usepackage{algorithmicx}
\usepackage{algpseudocode}
\usepackage{py2tex}

% \noindent globale
\setlength\parindent{0pt}
\setlength\parskip{2pt}

% \myul{}: underline + colore blu
\newcommand{\myul}[2][black]{\setulcolor{#1}\ul{\textbf{#2}}\setulcolor{black}}

% \cbox{}: un quadratino per scriverci dentro un carattere o per farci sopra una X
\newcommand\cbox{\fbox{\rule{0.1in}{0pt}\rule[-0.2ex]{0pt}{2ex}}}

% nuovo tipo di colonna per tabularx che "espande" come 'X' ma centra come 'c'
\newcolumntype{Y}{>{\centering\arraybackslash}X}

% aggiusta leggermente l'altezza delle righe in tabularx (necessario per le \cbox{})
\setlength{\extrarowheight}{1pt}
\renewcommand{\arraystretch}{1.1}
\renewcommand{\tabularxcolumn}[1]{m{#1}}

% nasconde i numeri di pagina
\pagestyle{empty}

% ambente all'interno del quale non è possible andare a pagina nuova
\newenvironment{absolutelynopagebreak}
  {\par\nobreak\vfil\penalty0\vfilneg
   \vtop\bgroup}
  {\par\xdef\tpd{\the\prevdepth}\egroup
   \prevdepth=\tpd}

% definisce l'ambiente "esercizio"
\newcounter{indicedellesercizio}[section]
\newenvironment{esercizio}[1]{
    \begin{absolutelynopagebreak}
    \vspace{0.3cm}
    \hrule height 2pt
    \vspace{0.1cm}
    \Large
    \refstepcounter{indicedellesercizio}
    \emph{\textbf{Esercizio \#\theindicedellesercizio}: la risposta corretta vale \textbf{#1} \ifstrequal{#1}{1}{punto}{punti}}
    \normalsize
    \vspace{0.3cm}\\
}{
\end{absolutelynopagebreak}
\vspace{0.3cm}}

\newcommand{\rispostaaperta}{}
\newcommand{\rispostachiusa}[4]{
    \begin{absolutelynopagebreak}
    \begin{enumerate}
        \item #1
        \item #2
        \item #3
        \item #4
    \end{enumerate}
    \emph{Inserisci nel form il numero della risposta (1, 2, 3 oppure 4).}
    \end{absolutelynopagebreak}
}
\newcommand{\sezione}[1]{
    \clearpage
    \begin{center}
        \Huge #1
    \end{center}
}
\newcommand{\sezionelogicomatematica}{\sezione{Esercizi di carattere logico matematico}}
\newcommand{\sezioneprogrammazione}{\sezione{Esercizi di programmazione}}
\newcommand{\sezionealgoritmi}{\sezione{Esercizi di carattere algoritmico}}

\begin{document}


% ============================================================================

\begin{center}
    \begin{minipage}{0.45\textwidth}
        \centering
        \includegraphics[width=\textwidth]{oii.pdf}\\
    \end{minipage}
\end{center}

\vspace{0.15cm}

\begin{center}
    \Huge
    Olimpiadi Italiane di Informatica
    \huge 2020 -- 2021
    \\
    \LARGE Selezione Scolastica -- 23 febbraio 2021
\end{center}

\vspace{1cm}

{
    \Large

    \begin{tabularx}{\textwidth}{|l|Y|}  % |l|Y| per rimettere le linee verticali
        \hline
        \textbf{Atleta} & ${nome}$ ${cognome}$ \\
        \hline
        \textbf{Codice Univoco} & \texttt{${codice_form}$} \\
        \hline
    \end{tabularx}
}

\begin{itemize}
    \item Nelle pagine seguenti troverai solo esercizi \emph{a risposta numerica},
          ovvero domande in cui è richiesto di calcolare un certo risultato
          numerico, intero maggiore o uguale a zero, che dovrai scrivere direttamente nel campo
          di testo apposito.

    \item Le risposte devono essere inviate tramite
          \href{${link_form}$${codice_form}$}{\color{blue}
          \myul[blue] {questo Google Form}}. Una volta aperto il link, controlla che
          il campo ``Codice Univoco'' sia correttamente compilato con il tuo codice
          univoco: se così non fosse, scrivilo tu,  facendo attenzione a copiarlo
          correttamente.

          Se il link non dovesse funzionare correttamente, ce n'è un altro a piè di pagina.%
          \footnote{\tiny\href{${link_form}$${codice_form}$}{\texttt{${link_form}$${codice_form}$}}}

    \item Il Google Form \textbf{verrà chiuso alle ore ${ora_fine}$ in punto}, quindi hai
          circa 90 minuti a tua disposizione. È possibile inviare il form più di una 
          volta: \textbf{verrà considerato solo l'ultimo invio}. Considera che una
          volta cliccato ``Invia'' è ancora possibile modificare la risposta cliccando
          sul link ``\emph{Modifica la risposta}''. Puoi modificare le tue risposte
          quante volte vuoi: \textbf{ti consigliamo vivamente di salvare spesso} le tue
          risposte, senza aspettare di aver finito tutte le domande. In questo modo non
          correrai il rischio che un problema di connessione a fine gara ti impedisca di
          venire valutato.

          \textbf{Nota bene:} Fai attenzione a non chiudere la pagina del Google Form,
          perché altrimenti potresti perdere l'accesso alle tue risposte precedenti.
          In ogni caso \textbf{ti consigliamo vivamente di scrivere le tue risposte su un
          foglio di carta}, in modo da poterle reinserire velocemente nel form in caso
          di necessità.

    \item La prova consiste di \textbf{5 esercizi a carattere logico
          matematico}, \textbf{7 esercizi di programmazione} e \textbf{8
          esercizi a carattere algoritmico}. Il tempo a disposizione per la
          prova è piuttosto limitato, per cui ti suggeriamo di non fermarti a lungo
          su un esercizio se non riesci a trovarne la soluzione, ed eventualmente
          riprenderlo in esame quando avrai terminato di rispondere a tutti gli esercizi
          successivi.

    \item Ad ogni esercizio è associato un punteggio di 2 o 3 punti, correlato al livello
          di difficoltà. Il punteggio è indicato all'inizio dell'esercizio. Per ogni risposta esatta
          viene assegnato il punteggio corrispondente; mentre per ogni risposta sbagliata
          o mancante vengono assegnati \textbf{0} punti.
\end{itemize}

% ============================================================================

${questions}$

${solutions}$

\end{document}
